% --------------------------------------------------------------
% This is all preamble stuff that you don't have to worry about.
% Head down to where it says "Start here"
% --------------------------------------------------------------
 
\documentclass[12pt]{article}

\usepackage[margin=1in]{geometry} 
\usepackage{amsmath,amsthm,amssymb}
\usepackage{CJKutf8}
\newcommand{\N}{\mathbb{N}}
\newcommand{\Z}{\mathbb{Z}}
\newcommand{\norm}[1]{\left\lVert#1\right\rVert}
\newenvironment{theorem}[2][Theorem]{\begin{trivlist}
\item[\hskip \labelsep {\bfseries #1}\hskip \labelsep {\bfseries #2.}]}{\end{trivlist}}
\newenvironment{lemma}[2][Lemma]{\begin{trivlist}
\item[\hskip \labelsep {\bfseries #1}\hskip \labelsep {\bfseries #2.}]}{\end{trivlist}}
\newenvironment{exercise}[2][Exercise]{\begin{trivlist}
\item[\hskip \labelsep {\bfseries #1}\hskip \labelsep {\bfseries #2.}]}{\end{trivlist}}
\newenvironment{answer}[2][Answer]{\begin{trivlist}
\item[\hskip \labelsep {\bfseries #1}\hskip \labelsep {\bfseries #2.}]}{\end{trivlist}}
\newenvironment{proposition}[2][Proposition]{\begin{trivlist}
\item[\hskip \labelsep {\bfseries #1}\hskip \labelsep {\bfseries #2.}]}{\end{trivlist}}
\newenvironment{corollary}[2][Corollary]{\begin{trivlist}
\item[\hskip \labelsep {\bfseries #1}\hskip \labelsep {\bfseries #2.}]}{\end{trivlist}}
\newenvironment{problem}[2][Problem]{\begin{trivlist}
\item[\hskip \labelsep {\bfseries #1}\hskip \labelsep {\bfseries #2.}]}{\end{trivlist}} 
\begin{document}
\begin{CJK}{UTF8}{bsmi}
% --------------------------------------------------------------
%                         Start here
% --------------------------------------------------------------
 
%\renewcommand{\qedsymbol}{\filledbox}
 
\title{Machine Learning HW1}%replace X with the appropriate number
\author{林子雋\\ %replace with your name
b04705003 資工三} %if necessary, replace with your course title
\date{} %without date
\maketitle
 
\begin{problem}{1. 記錄誤差值 (RMSE)(根據kaggle public+private分數),討論兩種feature的影響}
 %You can use theorem, proposition, exercise, or reflection here.  Modify x.yz to be whatever number you are proving
$ $\newline 
Delete this text and write theorem statement here.
\end{problem}
 
\begin{problem}{2. 將feature從抽前9小時改成抽前5小時,討論其變化} 
$ $\newline
\end{problem}
\begin{problem}{3. Regularization on all the weight with λ=0.1、0.01、0.001、0.0001,並作圖} 
$ $\newline
\end{problem}
\begin{problem}{4. 線性回歸問題,請寫下算式並選出正確答案}
$ $\newline
這個問題可以被公式化成 
	$$ Xw = y, \text{where }w = [w_{1}, w_{2}, ..., w_{n}] $$
令最佳解為 $\hat{w}$目標要最小化 $\norm{y - X\hat{w}}^2 $,已知若要最小化此方程式, $X\hat{w}$必須等於$y$在$X$的column space 上投影, 因此問題可以被formula成
	$$  X\hat{w} = \text{proj}_{column\ space\  of X}y $$
又知$x\hat{w}-y$的row向量與$X$中column向量正交,因此
	$$ X^{T}(X\hat{w}-y) = 0$$
得知
	$$ \hat{w} = (X^{T}X)^{-1}X^{T}y $$
答案為(c)
\end{problem}


% --------------------------------------------------------------
%     You don't have to mess with anything below this line.
% --------------------------------------------------------------
\end{CJK} 
\end{document}